% !TeX root = ../INTO-CPS-Manifesto.tex

\section{Related Work}\label{sec:related}

Extensive work has been carried out to identify the main concepts and essential challenges in co-simulation.
In this section, we review some of these works.

\cite{Trcka2007} reviews principles and implementation strategies of co-simulation applied to an HVAC system.
It provides multiple experiments showing how the stability and accuracy of the co-simulation is affected by the choice of those strategies.

The work in \cite{Hafner2017} exposes the disparity in terminology related to co-simulation (e.g., the term ``co-simulation'' is understood as ``cooperative simulation'', or ``coupled system simulation''), provides an in-depth discussion of the multiple concepts, and proposes a way to classify and structure co-simulation methods.
The authors propose the distinction by:
\begin{compactdesc}
\item[state of development:] the motivation being the use of co-simulation (e.g., optimise the simulation of a single model by partitioning, or couple the behaviour of wildly different subsystems);
\item[application field:] see, e.g., the application fields described in \cite{Gomes&17};
\item[model description:]  the kind of models being combined (e.g., Ordinary Differential Equations (ODEs), Differential-Algebraic Equations (DAEs), discrete event systems);
\item[numerical approach:] the kind of coupling algorithms employed; and
\item[interfaces:]  the nature of the physical interfaces between the systems being coupled.
\end{compactdesc}

Recognising that co-simulation is not a new concept and that it has been applied in wildly different fields, \cite{Gomes&18} reviewed co-simulation approaches, research challenges, and research opportunities.
They apply feature oriented domain analysis \cite{Kang1990} to help map the field.
The main result is a feature model that classifies the requirements of co-simulation frameworks and the participating simulators.
They conclude that the main research needs are: finding generic approaches for modular, stable, valid, and accurate coupling of simulation units; and finding standard interfaces for hybrid co-simulation.

With a focus on power systems, but still covering the fundamental concepts, \cite{Palensky2017} highlights the value of co-simulation for the analysis of large scale power systems.
In a tutorial fashion, it goes over the main concepts and challenges, providing a great introduction for new researchers in the field.

Recognising the research in co-simulation should be driven by both industry and academia, \cite{Schweiger2018a} reports on an empirical survey, given to both practitioners and academics.
The preliminary results 
%\claudio{(An extended version is being published at the American Modelica Conference)} 
corroborate the challenges pointed out in the surveys we referenced here.
Additionally, it becomes clear that co-simulation is being used without in-depth knowledge of the subject, which may lead to the improper use of the technique, as well as highlighting the need to develop more usable tools.

Finally, \cite{Gomes2018b} discusses the past and future of co-simulation, providing an historical overview of the topic, as well as possible research directions.

%\claudio{Should we survey also co-simulation frameworks, or are the above surveys on the topic enough?
%I guess that if someone wants to learn about co-simulation, the surveys are a better way to start\ldots}
