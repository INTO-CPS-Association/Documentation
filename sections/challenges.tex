\section{Challenges in Engineering CPSs}\label{sec:challenges}

%\fbox{John Fitzgerald - rough notes at this point}

The vision underpinning \into\ is that teams of developers from diverse disciplines and organisations are enabled to collaborate and converge more rapidly than today on system designs that perform optimally. Realising this vision requires methods and tools that allow each discipline and organisation to make its contribution without compromising its Intellectual Property (IP) or significantly altering its well-tried and established techniques. It should be possible to federate these diverse design artefacts to allow analy\-sis of the system-level consequences of well-founded design decisions made in any one domain, and the trade-offs between them. How, then, can we use semantically well-founded approaches to support such multidisciplinary design in a cost-effective way? This question poses several significant challenges. 
%\claudio{In the above paragraph, it's not clear why it needs to be semantically well founded? The rest is clear, but that part feels like we assume it is necessary.}

\subsection{Time to Market}

There is a clear need for model-based methods that permit early design space exploration, optimisation and experimentation without delaying the launch of a new product to the market. Important issues (that INTO-CPS brings solutions to) here are:

\begin{description}
\item[Cutting time to market for engineering CPSs:] In a highly active CPS marketplace, getting the right solution first time is essential. We believe that the interoperability of tools in the INTO-CPS tool suite enables a more agile close collaboration between stakeholders with diverse disciplinary backgrounds.
\item[Exploring large design spaces efficiently:] CPS design involves making design decisions in both the cyber and physical domains. Trade-off analysis can be challenging. Co-simulation enables the systematic exploration of large design spaces in the search for optimal solutions.
\item[Limiting expensive physical tests:] CPS development often relies on the expensive production and evaluation of a series of physical prototypes. Co-simulation enables users to focus on testing different models of CPS elements in a virtual setting, gaining early assessment of CPS-level consequences of design decisions.
\end{description}
%\claudio{The above list should focus on issues, and not on solutions. It presents a solution to two of the items. Also, the first item is not consistent with the other ones (the verb). Maybe ``Cutting the time to market''?}

\subsection{Diversity of Design Models}

Disciplines such as software, mechatronic and control engineering have evolved notations and theories that are tailored to their needs. It is undesirable to suppress this diversity by enforcing uniform general-purpose models~\cite{Fitzgerald&15,Larsen&16e}. The semantics of these notations and theories will have very different foundations in discrete or continuous domains. The goal, then, must be to support the effective federation of such highly diverse design models.

\subsection{Collaboration}

There is a clear need to provide mechanisms to support collaborative model-based engineering without compromising the independence of contributors. Important issues (that INTO-CPS brings solutions to) here are:
\begin{description}
\item[Avoiding vendor lock-in by open tool chain:] Some commercial solutions provide at least a part of the functionality provided by the INTO-CPS tool chain with a high level of interoperability. However, in particular for Small and Medium-sized Enterprises (SMEs), there is a risk of being restricted in the choice of specialist tools.
\item[Traceability and Provenance:] CPS development often relies on the expensive production and evaluation of a series of physical prototypes. Co-simulation enables users to focus on testing different models of CPS elements in a virtual setting, gaining early assessment of CPS-level consequences of design decisions.
\end{description}
%\claudio{Again, focus on issues.}

The foundations, methods and tools of CPS engineering should incorporate both the Discrete-Event (DE) models of computational processes, and the continuous-value and Continuous-Time (CT) formalisms of physical dynamics engineering. Our approach is to support the development of collaborative models containing DE and CT elements expressed in diverse notations, and to support their analysis by means of co-simulation based on a reconciled operational semantics of the individual notations' simulators \cite{Fitzgerald&14c}. This enables exploration of the design space and allows relatively straightforward adoption in businesses already exposed to some of these tools and techniques. 
%At the same time the semantic foundations are extended using Unifying Theories of Programming (UTP) to permit analysis using advanced meta-level tools that are primarily targeted towards academics and thus not considered as a part of the industrial INTO-CPS tool chain.

