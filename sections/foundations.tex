\section{The INTO-CPS Foundations}\label{sec:foundations}

\fbox{Jim Woodcock}

\fbox{JCPW: Add a short introduction}

\subsection{Foundations of the SysML profile for CPS modelling}

The INTO-CPS project proposes a novel technique for proof-based analysis of co-simulations that considers both architectural and behavioural properties of co-simulations. In D2.3a~\cite{INTO-CPS-D2.3a-2017}, the technique is illustrated by way of two case studies, one from from railways and another one from the area of smart buildings control. D2.2a~\cite{INTO-CPS-D2.2a-2016} instantiates the approach to robotic control.

\subsubsection{SysML}

The Systems Modelling Language (SysML)~\cite{OMGSysML2012} builds on the Unified Modelling Language (UML) to provide a general-purpose notation for systems engineering. SysML supports the modelling of cyber-physical systems (CPSs), which are designed to actively engage with the physical world in which they reside. They tend to be heterogeneous: their subsystems tackle a wide variety of domains (such as, mechanical, hydraulic, analogue, and a plethora of software domains) that mix phenomena of both continuous and discrete nature, typical of physical and software systems, respectively. Such systems are typically engineered using a variety of languages and tools that adopt complementary paradigms; examples are physics-related models, control laws, and sequential, concurrent, and real-time programs. This diversity makes CPS generally difficult to analyse and study.

\subsubsection{Co-simulation}

CPSs are often handled modularly to tackle this heterogeneity and complexity. To separate concerns effectively, the global model of the system is decomposed into subsystems, each typically focused on a particular phenomenon or domain and tackled by the most appropriate modelling technique. Simulation, the standard validation technique for CPS, is often carried out modularly also, using co-simulation~\cite{GomesTBLV2017,GomesTBLV2018}, the coupling of subsystem simulations. This constitutes the backdrop of the industrial Functional Mockup Interface (FMI) standard~\cite{FMIStandard2014,BromanBGLMTW2013,CavalcantiWA16} for co-simulation of components built using distinct modelling tools. The Functional Mock-up Interface (FMI) Standard has been proposed to address the challenge of interoperability, coupling different simulators and their high-level control components via a bespoke FMI API.

While co-simulation is currently the predominant approach to analyse CPS, INTO-CPS proposes a proof-based complementary technique that uses mathematical reasoning and logic. Simulation is useful in helping engineers to understand modelling implications and spot design issues, but cannot provide universal guarantees of correctness and safety. It is usually impossible to run an exhaustive number of simulations as a way of testing the system. For these reasons, it is often not clear how the evidence provided by simulations is to be qualified, since simulations depend on parameters and algorithms, and are software systems (with possible faults) in their own right.

Proof-based techniques, on the other hand, hold the promise of making universal claims about systems. They can potentially abstract from particular simulation scenarios, parametrisations of models, and interaction patterns used for testing. In traditional software engineering, they have been successfully used to validate the correctness of implementations against abstract requirements models~\cite{WoodcockLBF2009}. Yet, their application to CPS is fraught with difficulties: the heterogeneous combination of languages used in typical descriptions of CPS raises issues of semantic integration and complexity in reasoning about those models. The aspiring ideal of any verification technique is a compositional approach, and such approaches are still rare for CPS~\cite{NuzzoLFS2018}.

\subsubsection{The INTO-CPS approach to verification and co-simulation}

Our approach is to formally verify the well-formedness and healthiness of SysML CPS architectural designs as a prelude to co-simulation. The designs are described using INTO-SysML~\cite{INTO-CPS-D2.1a-2015}, a profile for multi-modelling and FMI co-simulation. The well-formedness checks verify that designs comply with all the required constraints of the INTO-SysML meta-model; this includes connector conformity, which checks the adequacy of the connections between SysML blocks (denoting components) with respect to the types of the ports being wired. The healthiness checks concern detection of algebraic loops, a feedback loop resulting in instantaneous cyclic dependencies; this is relevant because a desirable property of co-simulation, which often reduces to coupling of simulators, is convergence (where numerical analyses approximate the solution), which is dependent on the structure of the subsystems and cannot be guaranteed if this structure contains algebraic loops~\cite{KublerS2000,BromanBGLMTW2013}. The work in INTO-CPS demonstrates the capabilities of our verification workbench for modelling languages and engineering theories mechanised in the Isabelle proof assistant~\cite{NipkowK2014}, and the CSP process algebra~\cite{Hoare1985} with its accompanying FDR3 refinement-checker~\cite{GibsonRobinsonABR2016}.

Our technique is based on abstraction: we use a relational view of FMUs (functional mock-up units, the components in an FMI architecture) that abstracts from reactive behaviours as well as the API imposed by FMI. This allows us to focus on the fundamental properties of a co-simulation, while introducing details into the model view refinement that preserves those properties.

\subsubsection{Instantiation for robotics applications}

We have extended and restricted the INTO-SysML profile to deal with mobile and autonomous robotic systems. For modelling the controllers, we use RoboChart~\cite{LiMRCWT2017}. For modelling the robotic platform and the environment, we use Simulink~\cite{MathworksURL}. We have also given a behavioural semantics for models written in the profile using CSP. The semantics is agnostic to RoboChart and Simulink, and captures a co-simulation view of the multi-models based on the FMI API.

Our semantics can be used in two ways. First, by integration with a semantics of each of the multi-models that defines their specific responses to the simulation steps, we can obtain a semantics of the system as a whole. Such semantics can be used to establish properties of the system, as opposed to properties of the individual models. In this way, we can confirm the results of co-simulations via model checking or theorem proving, for example.

There are CSP-based formal semantics for RoboChart~\cite{MiyazawaCRLWT2016} and Simulink~\cite{MarriottZC2012,CavalcantiMW2013} underpinned by a precise mathematical semantics. Our next step is their lifting to provide an FMI-based view of the behaviour of models written in these notations. With that, we can use RoboChart and Simulink models as FMUs in a formal model of a co-simulation as suggested here, and use CSP and its semantics to reason about the co-simulation.

It is also relatively direct to wrap existing CSP semantics for UML state machines~\cite{DaviesC2003,RaschW2005} to allow the use of such models as FMUs in a co-simulation. In this case, traditional UML modelling can be adopted.

Secondly, we can use our semantics as a specification for a co-simulation. The work in~\cite{CavalcantiWA16} provides a CSP semantics for an FMI co-simulation; it covers not only models of the FMUs, but also a model of a master algorithm of choice. The scenario defined by an INTO-SysML model identifies inputs and outputs, and their connections. The traces of the FMI co-simulation model should be allowed by the CSP semantics of the INTO-SysML model.

There is no support to establish formal connections between a simulation and the state machine and physical models (of the robotic platform and the environment). The SysML profile proposed here supports the development of design models via the provision of domain-specific languages based on familiar diagrammatic notations and facilities for clear connection of models. Complementarily, as explained above, the semantics of the profile supports the verification of FMI-based co-simulations.  There are plans for automatic generation of simulations of RoboChart models~\cite{CavalcantiWA16}. The semantics we propose can be used to justify the combination of these simulations with Simulink simulations as suggested above.

\subsubsection{Future work}

We first suggest the development of a tool that supports the user of our technique in automatically generating the Isabelle/UTP architectural model, as well as a sketch of the behavioural model. The formal developer can use the sketch as a starting point, completing it with a detailed encoding of functional behaviours of FMUs. Secondly, elements of the refinement strategy from abstract into concrete FMU models ought be explored for a larger spectrum of case studies and examples, beyond the ones we presented in this report. Both these works could be tackled by the INTO-CPS Association.
