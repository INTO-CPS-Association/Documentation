%!TEX root = ../INTO-CPS-Manifesto.tex

\section{The INTO-CPS Tool Chain}\label{sec:toolchain}

\fbox{Christian K\"{o}nig with support from Etienne Brosse}

\fbox{make sure that this section fits to the other sections, incl. references}

\fbox{indicate maturity of the different tool features, in particular wrt the new ones}

This section discusses the interconnectivity of the different tools, and how the tools fit into the workflows and tasks that are covered by INTO-CPS. In particular, this section focuses on the features that were added during the INTO-CPS project, and in the framework of the INTO-CPS association. This section does \textit{not} describe all the tools in detail (here, the reader is referred to the different manuals, and to the User Manual (ref to the INTO-CPS tool manual).

[text on the overall tool-chain, refer to workflows, probably from previous section]

[Figure of the overall tool-chain]

\subsection{Modelio}

Modelio is an open-source modeling environment for various formalisms with many interfaces to import and export models. In the context of INTO-CPS, the support for SysML modelling is of primary importance, while Modelio can be extended with a range of modules to enable more modelling languages. In the terminology of the methods guidelines (e.g. \cite{INTOCPSD3.3a}), Modelio is a tool for the \textit{architectural modeling} and for \textit{requirements management}.

During the INTO-CPS project, a SysML profile was created, which is available as a module for Modelio 3.4 and 3.6 (see \url{http://forge.modelio.org/projects/intocps}). This INTO-CPS SysML profile extends Modelio with several functionalities that described in detail elsewhere (refs). Here, only those parts of the INTO-CPS SysML profile are discussed that add features for interconnectivity in the tool-chain.

To support the FMI multi-modelling approach, ModelDescription.xml files can now be imported into, and exported from a SysML Architectural Modeling diagram. Importing ModelDescription.xml files creates a SysML block with the corresponding flow ports and attributes, exporting them allows import in other modeling tools, such as those described below.

The Connections Diagram describes the signal flow between the different SysML blocks, which can each correspond to one FMU. Using the new INTO-CPS SysML profile, the Connections Diagram can be exported to an intermediary JSON format, which can then be imported by the INTO-CPS Application, to create a new Multi-Model.

%The INTO-CPS SysML Profile\\
%Import of ModelDescription.xml \\
%Export of ModelDescription.xml files (for modeling tools)\\
%Export of Connections Diagram (for the Application)\\
DSE modeling configuration\\
Requirements management\\

\subsection{OpenModelica}

MD.xml import\\
FMU export\\
Traceability support\\

\subsection{20-sim}

MD.xml import\\
FMU export\\
Traceability support\\

\subsection{20-sim4C}

\subsection{RT Tester}

Model checking\\
Test automation\\

\subsection{Overture}

MD.xml import\\
FMU export\\
Traceability support\\

\subsection{AutoFocus3}

\subsection{4Diac}

\subsection{The INTO-CPS Application}

Integration of the different artifacts\\ 
configuration and execution of Co-Simulation\\
Front-end for Test-automation and model checking\\
Viewer for traceability\\
