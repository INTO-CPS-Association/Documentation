%!TEX root = ../INTO-CPS-Manifesto.tex

\section{Introduction}\label{sec:intro}

\fbox{Peter Gorm Larsen}

% Freely adapted from my own text on the into-cps app paper
The design of cyber-physical systems (CPSs), systems where computing and
physical processes interact closely, involves the usage of results obtained
using a combination of different formalisms serving different engineering
disciplines. Compounding the variety of formalisms, the formalisms are usually
associated with different tools. Either as different versions, or supported by 
different entities.

In this manifesto we put forward the idea of combining different formalisms and
respective supporting tools by means of a tool chain. A tool chain allows the
practitioner to easily combine the results from the various fields during the
design process.  Our view is to keep the best of each of the tools and
integrating them by building an environment automating the tool chain
interaction, accessibility, and unifying the entry point to it. 

In such unified environment of several tools, users keep their usual patterns
of interaction with their familiar tool or tools available in the tool chain.
Instead of having to learn a new formalism\ldots


But, on the other hand, it is hard to master all the components of the tool
chain.  The mastering of tool chains involves the management of the whole set
of models and inputs/results for each of the tools. In addition, the conversion
of results, traceability of changes and keeping track of the user interactions
need also to be taken into account.  Moreover, given the evolving nature of
each of the tools, the task of managing the tool chain while ensuring the
dependencies of each of the tools and their interoperability becomes too
complex. 

Practice shows users are more receptive to a  push-button approach when it is
time to combine their results with the other tools.  

The INTO-CPS builds upon a frontend application which was developed following the
unified entry point idea.  The INTO-CPS application allows the user to fetch
the different tools, checkout model files from repositories, orchestrate the
several tools run, and organize the results/interactions.  This reduces the
previously mentioned mastering complexity demands to a push-button effort.
 


