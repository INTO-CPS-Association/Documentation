%% ----------------------------------------------------------------------
%% $Revision: 6391 $
%% $Id: ta-modelling-guidelines.tex 6391 2017-06-08 14:13:27Z oliver.moeller@verified.de $
%% $URL: https://svn.nfit.au.dk/svn/iha_pgl-INTOCPS/WP4/D4_3a_User_Manual/sections/ta-modelling-guidelines.tex $
%% ----------------------------------------------------------------------
%% @TABLE OF CONTENTS:                 [TOCD: 15:26 08 Jun 2017]
%%
%% ----------------------------------------------------------------------
%% @SPELL: british                      Thu Jun  8 16:12:01 2017
%% ----------------------------------------------------------------------
\subsection{Modeling Guidelines (for TA and MC purposes)}\label{sec:modeling.guidelines.TA.MC}
When creating a model for test automation (TA) or model checking (MC), it is
important to keep in mind that it will be input to a {\em solver} that
explores the state space of the model. 
It is not helpful to have a very precise state machine model which the solver
cannot handle.\footnote{\emph{I.\@e.\@}, resources on time, memory or user patience will
  be depleted before completion of the ``solving'' task.} 

A {\em ``good''} model tries to find a balance between precision and solver
time, \emph{i.\@e.\@}, it aims to represent all important logical aspects (that may
contain design errors,) but abstract away from details where the
implementation can be trusted to handle data correctly.
%
This section contains some guidelines on how to avoid common modelling pitfalls.
%
\paragraph{(G.1) Model independent parts as separate machines.}
When you can think of two parts of the system as individual entities, then
you should have one state machine for each part. This is much easier to
maintain than a state machine that ``merges'' two (or more) behaviours.
%
\paragraph{(G.2) Reset auxiliary variables.}
If you use ``auxiliary'' variables in your model (\emph{e.\@g.\@} to remember some
situation) then they should have a limited lifetime where they can be of
relevance.  Reset them (to ``0'') when they have fulfilled their purpose. This modelling
trick will not influence the relevant behaviour of your model, but make the solver's task easier.
%
\paragraph{(G.3) Keep the diameter of state machines small (if possible).}
The number of steps that are necessary to reach a certain situation can be a
serious limitation in finding solutions.  If your state machine includes long sequences of transitions that need to be
taken in order to reach a certain situation, consider simplifying it by
\begin{itemize}
\item breaking this state machine down into several state machines
\item ``merging'' several similar states along this path into a single, more
  abstract state (\emph{i.\@e.\@} make your model less precise)
\end{itemize}
%
\paragraph{(G.4) Use abstractions whenever specific (payload) data is of no interest.}
If the logical state depends on input data, try to model this as simply as
possible. Consider omitting data sanitation steps (like plausibility checks,
interpolation of values, data sanitation) that may exist in your specific
system but only serve as countermeasures to unreliable sensor inputs. In the
modelling world, inputs are always ``perfect'' and ``reliable''.
{\bf Example}: When modelling a protocol, avoid modelling the payload. Also, do not put payload and addressing (housekeeping) information in the
same variable.
%
\paragraph{(G.5) Try to think of ``logical states'' rather than of ``data driven states''.}
Variables can be used to ``encode'' the states of your system. Try to find a
balance where state machine states actually represent ``logical states'' (that
are distinguished by their behaviour). This will make the guard conditions
smaller and more maintainable.
%
\paragraph{(G.6) Keep a list of modelling parameters.}
Keep an overview on what immutable ``parameters'' your modelling depends on,
\emph{i.\@e.\@}, what values they have and where they are used.
This will make it easier to revise your model consistently when required.
%
\paragraph{(G.7) Dare to approximate complex values.}
In particular {\em continuous flows} may need to be broken down into manageable
components.
{\bf Example}: A physical accumulator can be approximated by a scalar value. 
Using a (scaled) integer instead of a floating-point number may be good enough and make the
task of the solver much easier.
%
\paragraph{(G.8) Dare to approximate complex behaviour.}
A common modelling mistake is to ``model as precise as possible''. 
In particular, capturing continuous states can be very expensive (in terms of
state space).  Whenever possible, you should consider simplifying things, even if this
means sacrificing precision. Remember that testing aims to find ``logical
errors'' and not to ``represent the most precise approximation of the system''.
{\bf Example}: When modelling a water tank with (continuous) in-flow and state
change, it may suffice to create a ``coarse'' model of it which changes state
only once per second, based in input value, water level, and maximal output
capacity. 
%
\paragraph{(G.9) Start with small versions of your system.}
Do not start modelling with a full-blown version of a complete system. Rather
start with one relevant component, say, one controller and explore testing a
small system
before moving to ``many controllers, many other components''.
%% -------------------------------------------------------------------------