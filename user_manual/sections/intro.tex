\section{Introduction}\label{sec:intro}
%This deliverable is the user manual for the INTO-CPS tool chain.
%
The tool chain supports the development and verification of Cyber-Physical Systems (CPSs) through collaborative modelling (co-modelling) and co-si\-mul\-at\-ion \cite{Gomes&17}.
%
Development of CPSs with the INTO-CPS technology proceeds with the development of constituent models using established and mature modelling tools.
%
Development also benefits from support for Design Space Exploration (DSE).
%
The analysis phase is primarily based on co-simulation of heterogeneous models compliant with version 2.0 of the Functional-Mockup Interface (FMI) standard for co-simulation \cite{FMIStandard2.0}.
%
Other verification features supported by the tool chain include hardware-\@ and software-in-the-loop (HiL and SiL) simulation and model-based testing through Linear Temporal Logic model checking.

All INTO-CPS tools can be obtained from
%
\begin{quote}
\url{http://into-cps-association.github.io}
\end{quote}
%
%
%
This is the primary source of information and help for users of the INTO-CPS tool chain.
%
The structure of the website follows the natural flow of CPS development with INTO-CPS, and serves as a natural aid in getting started with the technology.
%
In case access to the individual tools is required, pointers to each are also provided.

\textbf{Please note}:  This user manual assumes that the reader has a good understanding of the FMI standard.
%
We therefore strongly encourage the reader to become familiar with Section 2 of Deliverable 4.1d \cite{INTOCPSD41d} for background, concepts and terminology related to FMI.

The rest of this manual is structured as follows:
%
%
%
\begin{itemize}
%
\item  Section~\ref{sec:overview} provides an overview of the different features and components of the INTO-CPS tool chain.
%
\item  Section~\ref{sec:app} explains the different features of the main user interface of the INTO-CPS tool chain, called the INTO-CPS Application.
%
\item  Section~\ref{sec:SysML} explains the relevant parts of the Modelio SysML modelling tool.
%
\item  Section~\ref{sec:simulators} describes the separate modelling and simulation tools used to build and analyse the different constituent models of a multi-model.
%
\item  Section~\ref{sec:DSE} describes Design Space Exploration (DSE) for INTO-CPS multi-models.
%
\item  Section~\ref{sec:Verification} describes model-based test automation and model checking in the INTO-CPS context.
%
\item  Section~\ref{sec:traceability} describes traceability along the INTO-CPS tool chain.
%
\item  Section~\ref{sec:CodeGen} provides a short overview of code generation in the INTO-CPS context.
  %
\item  Section~\ref{sec:CoHLA} Provides information related to the Configuring HLA (CoHLA) domain specific language.
%
\item  Section~\ref{sec:issues} describes how issues with the INTO-CPS tool chain are reported and handled.
%
\item  Section~\ref{sec:conclusions} presents concluding remarks.
%
\item  The appendices are structured as follows:
%
\begin{itemize}
%
\item  Appendix~\ref{appendix:acronyms} lists the acronyms used throughout this document.
%
\item  Appendix~\ref{appendix:tools} gives background information on the individual tools making up the INTO-CPS tool chain.
%
\item  Appendix~\ref{appendix:principles} gives background information on the various principles underlying the INTO-CPS tool chain.
%
%\item Appendix E relating the INTO-CPS tool chain to the leading tools are deployed in practice, and their strengths and weaknesses
%
\end{itemize}
%
\end{itemize}
