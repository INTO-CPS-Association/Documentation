\section{Configuring HLA (CoHLA)}\label{sec:CoHLA}
The documentation concerning the Configuring HLA (CoHLA) domain specific language is located within the CoHLA-INTO-CPS directory.
It consists of three reports:
\begin{itemize}
  \item An introduction to CoHLA and relation with INTO-CPS \dash an extract is provided below\footnote{\url{https://github.com/INTO-CPS-Association/Documentation/blob/master/CoHLA-INTO-CPS/CoHLA\%20-\%20INTO-CPS.pdf}}.
  \item An installation manual\footnote{\url{https://github.com/INTO-CPS-Association/Documentation/blob/master/CoHLA-INTO-CPS/CoHLA\%20-\%20Installation\%20manual.pdf}}.
  \item A user manual\footnote{\url{https://github.com/INTO-CPS-Association/Documentation/blob/master/CoHLA-INTO-CPS/CoHLA\%20-\%20User\%20manual.pdf}}.
\end{itemize}
{\itshape
  Configuring HLA (CoHLA) is a domain specific language (DSL) that allows the user to quickly
  construct a co-simulation from a set of simulation models. In contrast with INTO-CPS, CoHLA only
  focuses on the construction of the co-simulation and a number of additions for the co-simulation instead
  of the full CPS design. Its goal is to minimise the effort that is required to construct a co-simulation, so
  that existing methodologies could easily adapt CoHLA for co-simulation construction. CoHLA uses the
  HLA standard for the coordination of the co-simulation. From a CoHLA co-simulation specification,
  source code is generated that can be used with an implementation of the HLA standard. Currently, only
  the open source HLA implementation OpenRTI is supported by CoHLA.}
